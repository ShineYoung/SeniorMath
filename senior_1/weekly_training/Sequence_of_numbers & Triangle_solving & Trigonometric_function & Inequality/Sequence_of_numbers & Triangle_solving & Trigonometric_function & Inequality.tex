% !TEX program = xelatex

\documentclass[17pt,twoside,space]{ctexart}
\usepackage{NEMT}
\usepackage{amsmath}
\usepackage{graphicx}
\usepackage{tikz}
\usepackage{amssymb}

\begin{document}\zihao{5}

\biaoti{2019年5月 高中一年级 周练}
\fubiaoti{高一数学}
% {\heiti 注意事项}:
% \begin{enumerate}[itemsep=-0.3em,topsep=0pt]
% \item 答卷前,考生务必将自己的姓名和准考证号填写在答题卡上。
% \item 回答选择题时,选出每小题答案后,用铅笔把答题卡对应题目的答案标号涂黑。如需改动,用橡皮擦干净后,再选涂其它答案标号。回答非选择题时,将答案写在答题卡上。写在本试卷上无效。
% \item 考试结束后,将本试卷和答题卡一并交回。
% 	请认真核对监考员在答题卡上所粘贴的条形码上的姓名、准考证号与您本人是否相符。
% \end{enumerate}

%%====================================================================
%%—————————————————————————————正文开始———————————————————————————————
%%====================================================================

\section{选择题:本大题共12小题,每小题5分,共60分。在每小题给出的四个选项中,只有一项是符合题目要求的。}

\begin{enumerate}[itemsep=0.2em,topsep=0pt]

% 选 1
\item 与$-765^\circ$终边相同的角为(\hspace{7pt})
\begin{tasks}(4)
	\task $405^\circ$ \task $665^\circ$ \task $-215^\circ$ \task $275^\circ$ 
\end{tasks}

% 选 2
\item 已知等比数列 $\{a_n\}$ 满足 $a_1 = 3, a_1 + a_3 + a_5 = 21$, 则 $a_3 + a_5 + a_7 =$(\hspace{7pt})
\begin{tasks}(4)
	\task $42$ \task $84$ \task $21$ \task $6$ 
\end{tasks}

% 选 3
\item 已知等差数列$\{a_n\}$ 一共有 $12$ 项, 其中奇数项之和为 $10$, 偶数项之和为 $22$, 则公差为(\hspace{7pt})
\begin{tasks}(4)
	\task $1$ \task $5$ \task $12$ \task $2$ 
\end{tasks}

% 选 4
\item 函数 $f(x) = \cos (2x - \frac{2\pi }{7})$ 的最小正周期为(\hspace{7pt})
\begin{tasks}(4)
	\task $\frac{\pi}{2}$ \task $2\pi$ \task $\pi$ \task $4\pi$ 
\end{tasks}


% 选 5
\item 已知 $\triangle ABC$ 中, $a = 1, b = \sqrt{3}, A = \frac{\pi}{6}$, 则 $B$ 等于(\hspace{7pt})
\begin{tasks}(4)
	\task $\frac{\pi}{3}$ \task $\frac{\pi}{6}$ 或 $\frac{5\pi}{6}$ \task $\frac{\pi}{3}$ 或 $\frac{2\pi}{3}$ \task $\frac{\pi}{6}$ 
\end{tasks}

% 选 6
\item 下列区间中, 使函数 $y=\sin x$ 为增函数的是(\hspace{7pt})
\begin{tasks}(4)
	\task $[\pi , 0]$ \task $[0, \pi]$ \task $[-\frac{\pi }{2}, \frac{\pi }{2}]$ \task $[\frac{\pi }{2}, \frac{3\pi }{2}]$ 
\end{tasks}


% 选 7
\item 在 $\triangle ABC$ 中,角 $A, B, C$ 所对应的边分别为 $a, b, c$, 若角 $A, B, C$ 依次成等差数列, 且 $a = 1, b = \sqrt{3}$, 则 $S_{\triangle ABC} =$(\hspace{7pt})
\begin{tasks}(4)
	\task $\frac{\sqrt{3}}{2}$ \task $\sqrt{2}$ \task $\sqrt{3}$ \task $2$
\end{tasks}

% 选 8
\item 函数 $y = \tan (x+\frac{\pi }{5})$ 的一个对称中心是(\hspace{7pt})
\begin{tasks}(4)
	\task $(0,0)$ \task $(\frac{4\pi }{5},0)$ \task $(\frac{\pi }{5},0)$ \task $(\pi , 0)$
\end{tasks}

% 选 9
\item 已知数列 $\{a_n\}$ 满足递推关系: $a_{n+1} = \frac{a_n}{a_n + 1}, a_1 = \frac{1}{2}$, 则 $a_{2017} =$(\hspace{7pt})
\begin{tasks}(4)
	\task $\frac{1}{2017}$ \task $\frac{1}{2019}$ \task $\frac{1}{2016}$ \task $\frac{1}{2018}$ 
\end{tasks}

% 选 10
\item 我国南宋著名数学家秦九韶发现了已知三角形三边求三角形面积的"三斜求积公式", 设 $\triangle ABC$ 三个内角 $A, B, C$ 所对的边分别为 $a, b, c$, 面积为 $S$, 则"三斜求积公式"为 $S = \sqrt{\frac{1}{4}[a^2c^2-(\frac{a^2+c^2-b^2}{2})^2]}$. 若 $ a^2\sin C = 24\sin A, a(\sin C - \sin B)(c+b) =(27-a^2)\sin A$, 则用"三斜求积公式"求得的 $S=$(\hspace{7pt})
\begin{tasks}(4)
	\task $\frac{15\sqrt{5}}{4}$ \task $\frac{15\sqrt{6}}{4}$ \task $\frac{15\sqrt{7}}{4}$ \task $\frac{3\sqrt{165}}{4}$ 
\end{tasks}

% 选 11
\item 设$a=0.6^{0.4}, b=0.4^{0.6}, c=0.4^{0.4}$, 则$a, b, c$的大小关系为(\hspace{7pt})
\begin{tasks}(4)
	\task $a<b<c$ \task $c<a<b$ \task $c<b<a$ \task $b<c<a$ 
\end{tasks}

% 选 12
\item 在数列$\{a_n\}$中, $a_1=1$, 当$n \geqslant 2$时, 其前 $n$ 项和为 $S_n$ 满足$S_n^2 = a_n(S_n-1)$, 设 $b_n=\log _2 \frac{S_n}{S_{n+2}}$, 数列$\{b_n\}$的前$n$项和为$T_n$, 则满足$T_n \geqslant 6$的最小正整数$n$是(\hspace{7pt})
\begin{tasks}(4)
	\task $11$ \task $9$ \task $12$ \task $10$ 
\end{tasks}

\end{enumerate}


% ======================================================================================================
\section{填空题:本题共4小题,每小题5分,共20分。}
% ======================================================================================================

\begin{enumerate}[itemsep=-0.3em,topsep=0pt,resume]%\setcounter{enumi}{12}

% 填 13
\item 在$\triangle ABC$中, $a = \sqrt{3}b, A=120^\circ$,则角 $B$ 的大小为\blank. 

% 填 14
\item 在一个塔底的水平面上某点测得塔顶的仰角为 $\theta$, 由此点向塔底沿直线行走了 $30m$, 测得塔顶的仰角为 $2\theta$, 再向塔底前进 $10m$, 又测得塔顶的仰角为 $4\theta$, 则塔的高度为\blank $m$.


% 填 15
\item 若数列$\{a_n\}$的前$n$项和 $S_n = n^2-10n (n\in \mathbb{N}_+)$, 则此数列的通项公式为\blank ; 数列$\{na_n\}$中数值最小的项是第\blank 项. 

% 填 16
\item 定义在 $\mathbb{R}$ 上的运算: $a \oplus b = ab+2a+b$, 则关于 $x$ 的不等式 $x \oplus (x-2) < 0$ 的解集为\blank . 

\end{enumerate}


% ======================================================================================================
\section{解答题:共70分。解答应写出文字说明、证明过程或演算步骤。}
% ======================================================================================================


\begin{enumerate}[itemsep=-0.3em,topsep=0pt,resume]%\setcounter{enumi}{17}

% 解 17
\item (10 分)\\
求解下列不等式的解集. 
\begin{enumerate}[itemsep=-0.3em,label={(\arabic*)},topsep=0pt,labelsep=.5em,leftmargin=1.7em]
	\item $-2x^4 - x^2 + 3 > 0$
	\item $x + \frac{1}{x} > 2$ 
	\item $(x+1)(x^2-2x-3)(x^2+3x-10) \leqslant 0$ 
	\end{enumerate}

	\vspace{180pt}

% 解 18
\item (12 分)\\
% Definition of circles
\def\firstcircle{(-0.4, 0) ellipse (19pt and 16pt)}
\def\secondcircle{(0.4, 0) ellipse (17pt and 15pt)}
\begin{minipage}[b]{0.65\linewidth}
	设全集为 $U=\mathbb{R}$, 集合 $A=\{x|-5 < x \leqslant 1\}, B=\{x|\frac{x+4}{x-7} < 0\}$.
	\begin{enumerate}[itemsep=-0.3em,label={(\arabic*)},topsep=0pt,labelsep=.5em,leftmargin=1.7em]
		\item 求如图阴影部分表示的集合; 
		\item 已知 $C = \{x|a+1 \leqslant x \leqslant 2a-1\}$, 若 $C \cup B = B$, 求实数 $a$ 的取值范围. 
		\end{enumerate}
	\end{minipage}
	\hfill
	\begin{minipage}[b]{0.35\linewidth}
		\begin{tikzpicture}
			\draw (-1.5, 1) rectangle (1.5, -1) \secondcircle node[right]{$B$};
			\node  at (-1.3, 0.8) {$U$}; 
			\begin{scope}
				\clip \firstcircle;
				\draw[fill, even odd rule] \firstcircle
                                     \secondcircle;
			\end{scope}
		\end{tikzpicture} 
	\end{minipage}

	
    
\newpage

% 解 19
\item (12 分)\\
设 $S_n$为正项数列 $\{a_n\}$ 的前 $n$ 项和, 满足 $2S_n = a_n^2 + n$. 
    \begin{enumerate}[itemsep=-0.3em,label={(\arabic*)},topsep=0pt,labelsep=.5em,leftmargin=1.7em]
	\item 求数列$\{a_n\}$的通项公式;
	\item 在等比数列 $\{b_n\}$ 中, $b_1=a_2, b_2=a_4, c_n=a_nb_n$, 求数列 $\{C_n\}$ 的前 $n$ 项和 $T_n$.
	\end{enumerate}

\vspace{220pt}

% 解 20
\item (12 分)\\
人工智能(Artificial Intelligence),英文缩写为 AI。它是研究、开发用于模拟、延伸和扩展人的智能的理论、方法、技术及应用系统的一门新的技术科学。科大讯飞信息科技股份有限公司是一家专业从事智能语音及语音技术研究、软件及芯片产品开发、语音信息服务的国家级骨干软件企业。讯飞公司现研发了一种新产品,固定成本7500元,每生产一台产品须增加投入 100 元,鉴于市场等多因素有总收入满足函数: $F(x) = \left\{\begin{array}{ll}
	400x-x^2, & 0 \leqslant x \leqslant 200\\
	40000, & x > 200
	\end{array}\right.$, 其中$y$为产品的每月产量, 利润=总收入-总成本.
	\begin{enumerate}[itemsep=-0.3em,label={(\arabic*)},topsep=0pt,labelsep=.5em,leftmargin=1.7em]
		\item 求利润表示为月产量的函数; 
		\item 求月产量为何值时,车间所获利润最大?并求出最大利润值. 
	\end{enumerate}

\newpage

% 解 21
\item (12 分)\\
已知 $f(x) = \vec{m} \cdot \vec{n}$, 其中 $\vec{m}=(2\cos x, 1), \vec{n}=(\cos x, \sqrt{3}\sin 2x) (x\in \mathbb{R})$. 
    \begin{enumerate}[itemsep=-0.3em,label={(\arabic*)},topsep=0pt,labelsep=.5em,leftmargin=1.7em]
		\item 求 $f(x)$ 的最小正周期及单调递增区间;
		\item 在 $\triangle ABC$ 中, $a, b, c$ 分别是角 $A, B, C$ 的对边, 若 $f(A)=2, b=1, \triangle ABC$ 面积为 $\frac{3\sqrt{3}}{2}$, 求边 $a$ 的长及 $\triangle ABC$ 的外接圆半径 $R$.
	\end{enumerate}

	\vspace{250pt}

% 解 22
\item (12 分)\\
已知正项数列$\{a_n\}$ 的前 $n$ 项和为 $S_n$, 首项 $a_1=1$,  点$P(a_n, a_{n+1}^2)$ 在曲线 $y = x^2+4x+4$ 上. 
    \begin{enumerate}[itemsep=-0.3em,label={(\arabic*)},topsep=0pt,labelsep=.5em,leftmargin=1.7em]
		\item 求 $a_n$ 和 $S_n$;
		\item 若数列 $\{b_n\}$ 满足 $b_1=17, b_{n+1}-b_n = 2n$, 求 $\frac{b_n}{\sqrt{S_n}}$ 最小时的 $n$ 值.
	\end{enumerate}

\end{enumerate}



%%%%%%%%%%%%%%%%%%%%%%%%%%%%%%%%%%%%%%%%%%%%%%%%%%%%%%%%%%%%%%%%%%%%%%%%%%%%%%%
%------------------------------------结束--------------------------------------
%%%%%%%%%%%%%%%%%%%%%%%%%%%%%%%%%%%%%%%%%%%%%%%%%%%%%%%%%%%%%%%%%%%%%%%%%%%%%%%
\clearpage

\end{document}
