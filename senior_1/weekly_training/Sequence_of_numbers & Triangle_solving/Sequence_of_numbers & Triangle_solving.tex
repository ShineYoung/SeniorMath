% !TEX program = xelatex

\documentclass[17pt,twoside,space]{ctexart}
\usepackage{NEMT}
\usepackage{amsmath}

\begin{document}\zihao{5}

\biaoti{2019年4月 高中一年级 练习题}
\fubiaoti{高一数学}
% {\heiti 注意事项}:
% \begin{enumerate}[itemsep=-0.3em,topsep=0pt]
% \item 答卷前,考生务必将自己的姓名和准考证号填写在答题卡上。
% \item 回答选择题时,选出每小题答案后,用铅笔把答题卡对应题目的答案标号涂黑。如需改动,用橡皮擦干净后,再选涂其它答案标号。回答非选择题时,将答案写在答题卡上。写在本试卷上无效。
% \item 考试结束后,将本试卷和答题卡一并交回。
% 	请认真核对监考员在答题卡上所粘贴的条形码上的姓名、准考证号与您本人是否相符。
% \end{enumerate}

%%====================================================================
%%—————————————————————————————正文开始———————————————————————————————
%%====================================================================

\section{选择题:本大题共12小题,每小题5分,共60分。在每小题给出的四个选项中,只有一项是符合题目要求的。}

\begin{enumerate}[itemsep=0.2em,topsep=0pt]

% 选 1
\item 设数列 $\{a_n\}$ 的前 $n$ 项和 $S_n=n^2$, 则 $a_8$ 的值为(\hspace{7pt})
\begin{tasks}(4)
	\task $15$ \task $16$ \task $49$ \task $64$ 
\end{tasks}

% 选 2
\item 已知数列 $\{a_n\}$ 为等差数列,其前 $n$ 项和为 $S_n$, 若 $a_3=6, S_3=12$, 则公差 $d$ 等于(\hspace{7pt})
\begin{tasks}(4)
	\task $1$ \task $\frac{5}{3}$ \task $2$ \task $3$ 
\end{tasks}

% 选 3
\item 已知等差数列 $\{a_n\}$ 的前 $n$ 项和为 $S_n$, 满足 $a_{2016}=S_{2016}=2016$, 则 $a_1$ 等于(\hspace{7pt})
\begin{tasks}(4)
	\task $-2017$ \task $-2016$ \task $-2015$ \task $-2014$ 
\end{tasks}

% 选 4
\item 在 $\triangle ABC$ 中,若 $a\sin A \sin B+b\cos^2A=\sqrt{3}a$, 则 $\frac{b}{a}=$(\hspace{7pt})
\begin{tasks}(4)
	\task $\sqrt{2}$ \task $\sqrt{3}$ \task $2\sqrt{2}$ \task $2\sqrt{3}$ 
\end{tasks}


% 选 5
\item 设等差数列 $\{a_n\}$ 的前 $n$ 项和为 $S_n$, 若 $S_3=9, S_6=36$, 则 $a_7+a_8+a_9$ 等于(\hspace{7pt})
\begin{tasks}(4)
	\task $63$ \task $45$ \task $36$ \task $27$ 
\end{tasks}

% 选 6
\item 在 $\triangle ABC$ 中,$a=5, b=7, c=8$, 则 $A+C=$(\hspace{7pt})
\begin{tasks}(4)
	\task $90^\circ$ \task $120^\circ$ \task $135^\circ$ \task $150^\circ$ 
\end{tasks}


% 选 7
\item 在 $\triangle ABC$ 中,若 $a=2b\cos C$, 则此三角形是(\hspace{7pt})
\begin{tasks}(2)
	\task 等腰直角三角形 \task 直角三角形 \task 等腰三角形 \task 等腰三角形或直角三角形
\end{tasks}

% 选 8
\item 已知数列 $\{x_n\}$ 满足 $x_1=1, x_2=\frac{2}{3}, \frac{1}{x_{n-1}}+\frac{1}{x_{n+1}}=\frac{2}{x_n} (n\geqslant 2)$, 则 $x_n$等于(\hspace{7pt})
\begin{tasks}(4)
	\task $(\frac{2}{3})^{n-1}$ \task $(\frac{2}{3})^n$ \task $\frac{n+1}{2}$ \task $\frac{2}{n+1}$
\end{tasks}

% 选 9
\item 已知等差数列 $\{a_n\}$ 中, $S_n$ 是它的前 $n$ 项和, 若 $S_{16}>0$, 且 $S_{17}<0$, 则当 $S_n$ 取最大值时 $n$ 的值为(\hspace{7pt})
\begin{tasks}(4)
	\task $16$ \task $8$ \task $9$ \task $10$ 
\end{tasks}

% 选 10
\item 若两个等差数列 $\{a_n\}$ 和 $\{b_n\}$ 的前 $n$ 项和分别是 $S_n, T_n$. 已知 $\frac{S_n}{T_n}=\frac{7n}{n+3}$, 则 $\frac{a_5}{b_5}$ 等于(\hspace{7pt})
\begin{tasks}(4)
	\task $7$ \task $\frac{2}{3}$ \task $\frac{27}{8}$ \task $\frac{21}{4}$ 
\end{tasks}

% 选 11
\item 在 $\triangle ABC$ 中,$AB=7, AC=6, M$ 是 $BC$ 的中点,$AM=4$, 则 $BC=$(\hspace{7pt})
\begin{tasks}(4)
	\task $\sqrt{21}$ \task $\sqrt{106}$ \task $\sqrt{69}$ \task $\sqrt{154}$ 
\end{tasks}

% 选 12
\item 在 $\triangle ABC$ 中,$a=x, b=2, B=45^\circ$, 若三角形有两解,则实数 $x$ 的取值范围为(\hspace{7pt})
\begin{tasks}(4)
	\task $x>2$ \task $2<x<2\sqrt{2}$ \task $x<2$ \task $2<x<2\sqrt{3}$ 
\end{tasks}

\end{enumerate}


% ======================================================================================================
\section{填空题:本题共4小题,每小题5分,共20分。}
% ======================================================================================================

\begin{enumerate}[itemsep=-0.3em,topsep=0pt,resume]%\setcounter{enumi}{12}

% 填 13
\item 在数列 $\{a_n\}$ 中, $a_1=1, a_2=5, a_{n+2}=a_{n+1}-a_n (n\in N_+)$, 则 $a_{2015}=$\blank. 

% 填 14
\item 在数列 $\{a_n\}$ 中, $a_1=2, a_{n+1}=a_n+ln\frac{n+1}{n}$, 则 $a_n=$\blank.

% 填 15
\item 在 $\triangle ABC$ 中,$a,b,c$ 互不相等,且 $a=4, c=3, A=2C$, 则 $b=$\blank. 

% 填 16
\item  在数列 $\{a_n\}$ 中, $a_1=2, a_n=\frac{2a_{n-1}}{a_{n-1}+2} (n\geqslant 2)$, 则 $a_n=$\blank. 

\end{enumerate}


% ======================================================================================================
\section{解答题:共70分。解答应写出文字说明、证明过程或演算步骤。}
% ======================================================================================================


\begin{enumerate}[itemsep=-0.3em,topsep=0pt,resume]%\setcounter{enumi}{17}

% 解 17
\item (10 分)\\
在 $\triangle ABC$ 中,$\cos(A-C)+\cos B=1, a=2c$, 求角 $C$.

	\vspace{180pt}

% 解 18
\item (12 分)\\
在 $\triangle ABC$ 中,$\sin^2B=2\sin A\sin C$
    \begin{enumerate}[itemsep=-0.3em,label={(\arabic*)},topsep=0pt,labelsep=.5em,leftmargin=1.7em]
	\item 若 $a=b$, 求 $\cos B$;
	\item 设 $B=90^\circ$, 且 $a=\sqrt{2}$, 求 $\triangle ABC$ 的面积. 
	\end{enumerate}
	
\newpage

% 解 19
\item (12 分)\\
数列 $a_n$ 的前 $n$ 项和为 $S_n$, 且 $S_n=n(n+1), n\in N_+$. 
    \begin{enumerate}[itemsep=-0.3em,label={(\arabic*)},topsep=0pt,labelsep=.5em,leftmargin=1.7em]
	\item 求数列$\{a_n\}$的通项公式;
	\item 若数列 $\{b_n\}$ 满足 $a_n=\frac{b_1}{3+1}+\frac{b_2}{3^2+1}+\cdots+\frac{b_n}{3^n+1}$, 求数列 $\{b_n\}$ 的通项公式.
	\end{enumerate}

\vspace{250pt}

% 解 20
\item (12 分)\\
在 $\triangle ABC$ 中,$a+c=6, b=2, \cos B=\frac{7}{9}$.
	\begin{enumerate}[itemsep=-0.3em,label={(\arabic*)},topsep=0pt,labelsep=.5em,leftmargin=1.7em]
		\item 求 $a,c$;
		\item 求 $\sin(A-B)$ 的值. 
	\end{enumerate}

\newpage

% 解 21
\item (12 分)\\
已知等差数列 $a_n$ 的前 $n$ 项和为 $S_n$, 且 $a_3+a_6=4, S_5=-5$. 
    \begin{enumerate}[itemsep=-0.3em,label={(\arabic*)},topsep=0pt,labelsep=.5em,leftmargin=1.7em]
		\item 求数列$\{a_n\}$的通项公式;
		\item 若 $T_n=|a_1|+|a_2|+|a_3|+\cdots+|a_n|$, 求 $T_5$ 的值和 $T_n$ 的表达式.
	\end{enumerate}

	\vspace{250pt}

% 解 22
\item (12 分)\\
已知数列 $\{\log_2a_n\}$ 是以 $1$ 为首项,$1$ 为公差的等差数列;数列 $\{b_n\}$ 满足 $b_1=1$, 且 $b_{n+1}=2b_n+2a_n (n\in N_+)$. 
    \begin{enumerate}[itemsep=-0.3em,label={(\arabic*)},topsep=0pt,labelsep=.5em,leftmargin=1.7em]
		\item 证明:数列 $\{\frac{b_n}{a_n}\}$ 为等差数列;
		\item 若对任意 $n\in N_+$, 不等式 $(n+2)b_{n+1}\geqslant \lambda b_n$ 总成立,求实数 $\lambda$ 的最大值.
	\end{enumerate}

\end{enumerate}



%%%%%%%%%%%%%%%%%%%%%%%%%%%%%%%%%%%%%%%%%%%%%%%%%%%%%%%%%%%%%%%%%%%%%%%%%%%%%%%
%------------------------------------结束--------------------------------------
%%%%%%%%%%%%%%%%%%%%%%%%%%%%%%%%%%%%%%%%%%%%%%%%%%%%%%%%%%%%%%%%%%%%%%%%%%%%%%%
\clearpage

\end{document}
